\documentclass[11pt]{article}   % tipo de documento e tamanho das letras

% os seguintes pacotes estendem a funcionalidade básica:
\usepackage[a4paper, total={16cm, 24cm}]{geometry} % tamanho da pagina e do texto
\usepackage[portuguese]{babel}  % traduz para portugues
\usepackage{graphicx}           % graficos
\usepackage{amsmath}            % matematica
\usepackage{tikz}               % diagramas
\usepackage{booktabs}           % tabelas com  melhor aspecto
\usepackage{hyperref}           % links para partes do documento ou para a web
\usepackage{listings}           % incluir codigo
    \renewcommand\lstlistingname{Listagem}  % Listing em portugues
    \lstset{numbers=left, numberstyle=\tiny, numbersep=5pt, basicstyle=\footnotesize\ttfamily, frame=tb,rulesepcolor=\color{gray}, breaklines=true}

% ----------------------------------------------------------------------------
\title{Relatório de ASC1}
\author{Mickey Mouse, nº 11111 \\ Donald Duck, nº 22222}
\date{\today}
% ----------------------------------------------------------------------------
\begin{document}
\maketitle

\abstract{O sistema \LaTeX permite escrever slides, artigos, relatórios e livros completos com grande qualidade tipográfica. Este texto ilustra alguns casos comuns de utilização.}


%===========================================================================
\section{Instalação}
\label{sec:instalacao}

O \LaTeX pode ser instalado em qualquer computador e está disponível nos sistemas operativos mais comuns (Linux MacOS, Windows, FreeBSD).

\begin{itemize}
    \item Em Linux e FreeBSD está disponível a distribuição TexLive que pode ser instalada com o gestor de pacotes do sistema operativo, por exemplo:
    \begin{itemize}
        \item \lstinline{sudo apt-get install texlive} (debian, mint, ubuntu)
        \item \lstinline{sudo pkg install texlive} (FreeBSD)
    \end{itemize}

    \item Em MacOS pode instalar-se de várias formas, mas a recomendada é a distribuição MacTex disponível em \url{http://www.tug.org/mactex/}
    \item Em Windows está disponível a distribuição \url{https://miktex.org}.
\end{itemize}

\section{Como usar}

Dois passos:
\begin{enumerate}
    \item Escrever o texto num ficheiro com extensão \lstinline{.tex} num editor de texto simples (e.g., geany) ou num editor \LaTeX (e.g. Kile, TexShop).
    \item Compilar usando o terminal com o comando \lstinline{xelatex ficheiro.tex} ou, no caso de editores dedicados, usando as funções disponíveis nos menus ou barra de ferramentas.
\end{enumerate}
Após a compilação sem erros, são gerados vários ficheiros sendo um deles o PDF final do documento.

\section{Como organizar o texto}

Ao contrário dos processadores de texto como o Word, onde é necessário formatar o texto, em latex não se formata o texto. Em vez disso, apenas se dá informação semântica. Por exemplo, o que é o título, autor, nome de uma secção, etc. É o sistema latex que decide qual o aspecto, onde são feitas as quebras das páginas, onde são colocadas as figuras, etc. Devemos, em geral, deixar prevalecer a opinião dele e não nos preocupar com o aspecto do documento.

As secções seguintes mostram como se podem incluir vários tipos de informação.

%===========================================================================
\section{Figuras}
\label{sec:figuras}

Na figura~\ref{fig:fluxo} está um fluxograma. Esta é obtida de um ficheiro PNG gerado com outro programa (visio, inkscape, draw.io, etc) e que é incluído no documento durante a compilação. Os ficheiros JPEG e PNG têm sempre um aspecto esbatido quando se faz zoom. É preferível incluir imagens geradas em PDF.
\begin{figure}
    \centering
    \includegraphics[scale=0.75]{ficheiro.png}
    \caption{Fluxograma obtido de um ficheiro PNG. As linhas ficam esbatidas com PNGs e JPEGs, mas com imagens em PDF ficam bem.}
    \label{fig:fluxo}
\end{figure}

Embora a figura tenha sido incluída entre este parágrafo e o anterior, a figura ``flutua'' para onde o \LaTeX~entender. Devemos respeitar a opinião dele, que normalmente é colocar as figuras no início das páginas. No texto devemos fazer referência à figura com o comando~\lstinline{\ref}.

%===========================================================================
\section{Fórmulas}
\label{sec:formulas}

As raízes do polinómio $ax^2+bx+c = 0$ são dadas por
\begin{equation}\label{eq:formula_resolvente}
    x = \frac{-b \pm \sqrt{b^2 - 4ac}}{2a}.
\end{equation}
A equação~\eqref{eq:formula_resolvente} chama-se \emph{fórmula resolvente}.

Um sistema linear
\[
    \begin{cases}
        ax + by = c \\
        dx + ey = f
    \end{cases}
\]
pode ser escrito de forma matricial como
\[
    \begin{bmatrix}
    a & b \\
    c & d \\
    \end{bmatrix}
    \begin{bmatrix}
    x \\
    y \\
    \end{bmatrix}
    =
    \begin{bmatrix}
    c \\
    f \\
    \end{bmatrix}.
\]

A equação mais bonita da matemática é a identidade de Euler
\begin{equation}
    e^{\pi i} + 1 = 0
\end{equation}
pois inclui exactamente um exemplar de cada uma das principais constantes ($0$, $1$, $\pi$, a constante de Neper $e$ e o número imaginário $i$) e das principais operações (adição, multiplicação e potência).

%===========================================================================
\section{TikZ}
\label{sec:tikz}

O TikZ permite fazer diagramas directamente em \LaTeX. Certas coisas fazem-se facilmente, outras nem por isso.
Devido à complexidade, vale a pena aprender apenas no final, quando já se domina bem o restante \LaTeX.
\begin{center}
\begin{tikzpicture}
    \node[circle, draw=blue, ultra thick, fill=green] (a) {A};
    \node[right of=a, node distance=3cm, fill=black, text=green] (b) {$\displaystyle\sum_{n=0}^{+\infty}n$};
    \draw[->, very thick] (a) to[out=-90,in=90] node[above] {$\sqrt{2}$} (b);
\end{tikzpicture}
\end{center}

%===========================================================================
\section{Listagens}
\label{sec:listagens}

Também há possibilidade de incluir código em várias linguagens.

\begin{lstlisting}[language=Python,caption=Código em Python.]
    def my_fun(a):
        b = 0
        for x in range(a):
            b += x              # this is a comment
        return b
\end{lstlisting}

%===========================================================================
\section{Tabelas}
\label{sec:tabelas}

O pacote booktabs permite melhorar o aspecto das tabelas.

\begin{center}
    \begin{tabular}{r|ccc}
    \toprule
      & A & B & C \\
    \midrule
    1 & 0.1 & 0.333 & 3.14 \\
    2 & $\sqrt{2}$ & $\pi$ & $1/3$ \\
    3 & $-\infty$ & $\xi$ & $\Xi$ \\
    \bottomrule
    \end{tabular}
\end{center}

%===========================================================================
\section{Conclusão}
\label{sec:conclusao}

Dadas as inúmeras vantagens deste sistema face aos processadores de texto convencionais, não há motivo não usar o \LaTeX na escrita de relatórios.

\end{document}